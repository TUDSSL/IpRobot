\chapter{Introduction}
\label{chp:introduction}

% printer: Tethered flight control of a small quadrotor robot for stippling \cite{galea_iros_2017}

%Example of energy harvesting: Prolonged energy harvesting for ingestible devices~\cite{plonski_tranro_2016}
%Drug delivery

% Estabish your territory

Miniature robots with limited capabilities can work together in "swarms" to achieve more than an individual could by itself, but are still far from being applicable in real world applications~\cite{barca_sekercioglu_2013}.
The future potential of swarms is widely recognized to have applications in surveillance, search and rescue, and exploration.
Swarms have been proposed as a new form of user interface, robots can display information and interact with users on tabletops~\cite{legoc_uist_2016}, unmodified clothing~\cite{dementyev_uist_2016} and can be an educational toy for kids~\cite{sony_toio_2017}.
\hfill \break
%TODO screenshots of robots to support statements

% Establish nice
% Motivation for this work: Why do we want to remove the battery?

However, one of the fundamental issues that needs to be addressed before swarm robotics can advance is related to energy.
Small lithium batteries are currently powering the robots and limit their operation time to only a few hours. 
The energy density of batteries has improved less than one order of magnitude since 1945, while in comparison the energy efficiency of computing has improved 12 orders of magnitude~\cite{patel_pvc_2017}.
The last major advancement in battery technology is 25 years old and came with the introduction of Li-ion batteries.
Additionally, any new big improvements in energy density of batteries is not likely to happen anytime soon, as new battery technologies are often overhyped and slow to emerge~\cite{zachary_spec_2016}.
\hfill \break

%TODO introduce current research

% Propose a solution: Energy harvesting
% And discuss current solutions to autonomus operation / battery replenishment

Stable energy supplies are required to allow long term persistent operation of robots.
Relaxing this constraint by allowing a transient supply of energy could be a potent area of research.
Wireless sensor nodes that rely on energy harvesting have already been emerging, some are highly energy constrained but fully eliminate the need for batteries. %TODO cite
In contrary, different replenishment methods for robot batteries are currently used, robots can be moved to a charging station leaving them nonoperational while recharging or the battery could be replaced, resulting in a high strain on maintenance.

% 4 lines left!
\newpage

\section{Problem statement}

% What why how

% Intermittently powered robots, that purely rely on harvested power are currently non existent.

Replacing a battery with an energy harvesting system could make a robot more self sufficient and energy-autonomous. 
This however introduces a new phenomenon to take into account; the intermittent availability of energy produces frequent power interrupts.
The possibility of sudden power loss is currently not considered when developing control software for a robot, and for this reason can not be used without adaption.
\hfill \break

Secondly, sensors and control techniques used may not be found applicable due to energy budget constrains or their inability to be interrupted by a loss of power. %TODO cite?
%TODO resolve big jump
Different locomotion types are used and may not all be reliable and/or accurate under the frequent power interruption.
Therefore the research question this work addresses is:

\begin{center}
	\textit{How to enable a transiently powered robot to operate autonomously?}
\end{center}

\section{Contributions}
Current robots assume that a task can only be completed if sufficient energy is left in the batteries, which also limits their operation time. 
This research will explore the feasibility of a miniature battery-less robot, allowing persistent operation while being supplied by a small and intermittent source of energy.
%This is the first study of transiently powered robots, unaware of any platforms being implemented and are operating under the constant threat of power loss. 
The list of contributions is as follows:

\begin{enumerate}

%\item A simple model is developed showing the relation between the energy stored, weight of the robot, frequency of power cycles and distance covered with a single charge.

\item Design of a battery-less robot that purely operates from harvested energy, with basic capabilities allowing autonomous operation.

%TODO state results of implementation in contribution
\item Implementation of a control algorithm, that allows the robot to finish a movement across power cycles.

%\item Evaluation of the battery-less robot compared to a battery powered robot in terms of, weight, speed and accuracy of movement.


\end{enumerate}


\section{Thesis Outline}
%What will be discussed in the coming chapters


\vspace{1\baselineskip}

\noindent
TODO ORGANISATIONAL DESCRIPTION OF THESIS

