\chapter{Introduction}
\label{chp:introduction}

\section{Motivation}

what kind of small robots are there
- Nanobots
- self asembling bots
- robot platforms for swarm





There is an increasing research interest in small autonomous robots, since they have the potential to 



Gathering information in case of disaster
Mobile sensing for hard to reach or inspect places
Collaboration increase fault tollerance


Robots need to explore their surroundings, sense, process and react accordingly.	


Small sizes allow robots to work in areas that would otherwise be unreachable. 
However these applications will require new sensing, control and navigation strategies to make use of the increasingly resource constrained robots.

Small robotic platforms still rely on batteries as a source of power, since electric motors consume considerably more energy compared to computation and sensing.
The currently most commonly used lithium-ion batteries, have a high energy density but still limit the operation time of the robot.
When the voltage of the battery drops below a certain level, indicating that the battery is almost empty, the robot needs to move to a charging station before it runs out of energy.
Another option could be to replace the battery, but this would still require the robot to return to a station.
Replacing the battery with an energy harvesting system would make the robot energy-autonomous. 

% what are the challenges???

\section{Problem statement}

% what is the goal??
Let's say you want to have it as a backup system.
Or it's a new approach to current robotic platforms.

Challenge:
Current software used for robots with a stable power supply cannot be ported directly to this new platform.
So what changes are required?

Intermittently powered robots, that purely rely on harvested power are currently non existent.
The intermittent availability of energy creates new challenges in control, navigation and actuation. 
This research will explore the feasibility of an autonomous battery-less robot, focusing on the development of such a robot and especially porting a complex task like navigation to it. 
Therefore the main question this work will try to answer is: \\
\textit{How to make a transiently powered robot navigate autonomously?}

\section{System Description}



\section{Contributions}

New approach to supplying energy to the robot which adds new challenges in computation

* first study of intermittently powered robots, unaware of any of such platforms besides hypothetical nanobots that supposed to be harvesting energy from movement or organic stuff.

* Design of a small robot with accurate locomotion, basic sensing and communication.

* Implementation of a complex task like navigation to this intermittently powered robot

* Evaluation of the performance of this new approach compared to a battery powered robot.

\section{Outline}
What will be discussed in the coming chapters


\vspace{1\baselineskip}

\noindent
TODO ORGANISATIONAL DESCRIPTION OF THESIS

