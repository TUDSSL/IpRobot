\chapter{Introduction}
\label{chp:introduction}

% printer: Tethered flight control of a small quadrotor robot for stippling \cite{galea_iros_2017}

% Estabish your territory

Small robots with limited capabilities can work together in "swarms" to achieve more than an individual could by itself, but are still far from applicable in real world applications~\cite{barca_sekercioglu_2013}.
The future potential of swarms is widely recognized to have applications in surveillance, search and rescue, and exploration.
Swarms have been proposed as a new form of user interface, robots can display information and interact with users on tabletops~\cite{legoc_uist_2016}, unmodified clothing~\cite{dementyev_uist_2016} and can be a educational toy for kids~\cite{sony_toio_2017}.
\hfill \break
% Establish nice
% Motivation for this work: Why do we want to remove the battery?

However, one of the fundamental issues that needs to be addressed before swarm robotics can advance is related to energy, while robots rely on small lithium batteries limiting their operation time to only a view hours. 
The energy density of batteries has improved less than 1 order of magnitude since 1945, while the energy efficiency of computing has improved 12 orders of magnitude~\cite{patel_pvc_2017}.
The last major advancement in battery technology is 25 years old and came with the introduction of Li-ion batteries.
Additionally, any new big improvements in energy density of batteries is not likely to happen anytime soon, as new battery technologies are often overhyped and slow to emerge~\cite{zachary_spec_2016}.
\hfill \break

%TODO introduce current research

% Propose a solution: Energy harvesting
% And discuss current solutions to autonomus operation / battery replenishment

A stable energy supply is a current requirement to allow long term persistent autonomous operation of swarm robots.
Relaxing this constraint by allowing a more transient supply of energy could be a potent area of research.
Wireless sensor nodes that rely on energy harvesting have already been emerging, and they fully eliminate the need for batteries.
In contrary, when the batteries of current swarm robots are depleted different methods are used to replenish the batteries, the robots could be moved to a charging station leaving them nonoperational while recharging or the battery could be replaced, resulting on a high strain on maintenance.

\newpage

\section{Problem statement}

%TODO add Challenges


Replacing the battery with an energy harvesting system would make the robot energy-autonomous. 
%So what changes are required?
All the previous work does not take any intermittentcy into account and assumes that an operation is only possible when there is sufficient energy to complete a task.
Current software used for robots with a stable power supply cannot be ported directly to this new platform.
This research will explore the feasibility of a battery-less robot, allowing persistent operation while having a very small and intermittent source of energy.
% Intermittently powered robots, that purely rely on harvested power are currently non existent.
The intermittent availability of energy creates new challenges in control, navigation and actuation. 
This research will explore the feasibility of an autonomous battery-less robot, focusing on the development of such a robot. Therefore the main question this work will try to answer is:

\begin{center}
	\textit{How to enable a transiently powered robot navigate autonomously?}
\end{center}

\section{Contributions}

\begin{enumerate}

\item First study of transiently powered robots, unaware of any of such platforms besides hypothetical nanobots that supposed to be harvesting energy from movement or organic stuff.

\item A battery-less robot design, where the robot purely operates from harvested energy. Which includes computation, sensing and short range communication.

\item Implementation of a complex task like navigation to this intermittently powered robot

\item An evaluation of the battery-less robot compared to a battery powered robot in terms of, weight, speed of movement and navigation accuracy.

% 

\end{enumerate}


\section{Thesis Outline}
%What will be discussed in the coming chapters


\vspace{1\baselineskip}

\noindent
TODO ORGANISATIONAL DESCRIPTION OF THESIS

