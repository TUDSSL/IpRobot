\chapter{Introduction}
\label{chp:introduction}

%TODO estabish your territory

% Start with a audiance grabbing first sentence + include ref

% Applications

% Hypothetical applications:
% - nuclear chemical and bioogical attack detection, battlefield surveillance, space exploration, polution detection and search and resque.

% studying swarm behavior

% user interfaces

% Mobile Wearables 

% toys for kids
% printer: Tethered flight control of a small quadrotor robot for stippling \cite{galea_iros_2017}

There is an increasing research interest in small autonomous robots, since they have the potential to 



Gathering information in case of disaster
Mobile sensing for hard to reach or inspect places
Collaboration increase fault tolerance
Robots need to explore their surroundings, sense, process and react accordingly.

Small sizes allow robots to work in areas that would otherwise be unreachable. 
However these applications will require new sensing, control and navigation strategies to make use of the increasingly resource constrained robots.
	
%TODO establish nice
% Motivation for this work: Why do we want to remove the battery?

One of the biggest problems in designing a practical Swarm robotics system for real world applications is energy related, as the whole system may shut down if energy is depleted~\cite{barca_sekercioglu_2013}.

Current swarm robotic platforms rely on Li-ion batteries as a source of power, while electric motors can consume considerably more energy compared to computation. % and sensing.

"Figure 1. The energy efficiency of computing has improved by 12 orders of magnitude since 1945. In the same time period, battery energy density has improved by less than 1 order of magnitude. The most dramatic improvements in battery density have been in the last 20 years with Li-ion batteries."
\cite{patel_pvc_2017}

"Low-hanging fruit is nowhere to be seen in fields as crucial as digital electronics, biomedical devices, or space technology."
\cite{zachary_spec_2016}


%TODO indroduce current research

% Propose a solution

% Current solutions to autonomus operation / battery rephisment

When the voltage of the battery drops below a certain level, indicating that the battery is almost empty, the robot needs to move to a station to recharge or swap it's battery before it runs out of energy.

Replacing the battery with an energy harvesting system would make the robot energy-autonomous. 


The current robotic community assumes that a stable energy supply is a requirement to allow long term persistent autonomous operation of robots.
Batteries are generally the stable energy source of choice, while different solutions have been proposed to recharge them.
Other solutions require the robot to be in close proximity to an energy source.

% Something about energy harvesting






%The currently most commonly used lithium-ion batteries, have a high energy density but still limit the operation time of the robot.

\section{Problem statement}

%TODO add Challenges

So what changes are required?
All the previous work does not take any intermittentcy into account and assumes that an operation is only possible when there is sufficient energy to complete a task.
Current software used for robots with a stable power supply cannot be ported directly to this new platform.
This research will explore the feasibility of a battery-less robot, allowing persistent operation while having a very small and intermittent source of energy.
% Intermittently powered robots, that purely rely on harvested power are currently non existent.
The intermittent availability of energy creates new challenges in control, navigation and actuation. 
This research will explore the feasibility of an autonomous battery-less robot, focusing on the development of such a robot.

%and especially porting a complex task like navigation to it. 

Therefore the main question this work will try to answer is:

\begin{center}
	\textit{How to enable a transiently powered robot navigate autonomously?}
\end{center}

\section{Contributions}

\begin{enumerate}

\item First study of transiently powered robots, unaware of any of such platforms besides hypothetical nanobots that supposed to be harvesting energy from movement or organic stuff.

\item A battery-less robot design, where the robot purely operates from harvested energy. Which includes computation, sensing and short range communication.

\item Implementation of a complex task like navigation to this intermittently powered robot

\item An evaluation of the battery-less robot compared to a battery powered robot in terms of, weight, speed of movement and navigation accuracy.
\end{enumerate}


\section{Thesis Outline}
%What will be discussed in the coming chapters


\vspace{1\baselineskip}

\noindent
TODO ORGANISATIONAL DESCRIPTION OF THESIS

