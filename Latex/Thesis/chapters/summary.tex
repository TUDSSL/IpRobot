\chapter{Summary}
\label{chp:summary}

%This thesis is surmised by a conclusion in Section \ref{sec:conclusion} and in Section \ref{sec:limitations_future_work} possibilities for future work will be presented.

\section{Conclusion}
\label{sec:conclusion}

In this thesis, a transiently-powered robot is designed and implemented that uses light as a source of energy.
Two geared dc motors have been chosen to provide locomotion to the robot, while the current consumed by the motors is limited using a microcontroller.
A controller is designed that allows controlled straight and curved movements and by employing a simple check pointing method and storing specific values in non-volatile memory a movement can be completed across multiple power cycles.
Multiple movements have been executed and recorded using a overhead camera.
Tracking software is used to verify the the accuracy of movement by comparing a transiently-powered robot to its battery powered comparable.
By decreasing the interrupt period a lower threshold of 0.3\,s has been found where the robot started to show uncontrolled behavior, i.e. significant drift to the side of the weakest motor.
While the amount of time that power is available might not be sufficient for the motors to reach their steady state speed, making controlled movements by linear motor control impossible.
Using artificial power interrupts, the results show that decreasing the interrupt period towards the lower threshold results in more horizontal deviation in case of straight movements.
For curved movements the results show a decreased average deviation from the fitted circle by increasing the target duty cycle.
The solar powered robot outperforms the battery-powered robot in terms of accuracy, as the additional weight and weight distribution of the solar panel which has a positive effect on the accuracy of movement of the robot.

%Section \ref{sec:controlled_movements} shows that the transiently powered robot is able to execute an instructed motion with similar accuracy compared to a battery powered equal. 
%The distance covered by a robot with frequent power interrupts in a certain amount of time is smaller, i.e the average speed is lower due to frequent acceleration from a standstill.

\section{Future Work}
\label{sec:limitations_future_work}

The current robot has very limited capabilities, which should be easy to extend by using more features from the WISP or by building a custom microcontroller board.
Additional features that could be implemented in feature work are:

\begin{itemize}

\item \textbf{Speed feedback} 
In order to move a certain distance with a higher accuracy, an energy efficient speed feedback method is required.
One option is to add local speed sensing or external position feedback could be supplied to the robot.

%One option could be to determine motor speed by measuring the Back-EMF produced by the motor, as it is proportional to the motors revolutions per minute~\cite{precision_backemf_2017}.
%Another method could use wheel encoders without an active light source and instead use ambient light, already available in abundance because it is used as the source of energy.

\item \textbf{Lack of communication}
At this moment communication with an external host is not implemented.
The backscatter communication channel of the WISP could be used.
While power cycles are less frequent when compared to the RFID, due to relative big supercapacitor, they still pose a challenge in externally controlling the robot.

\item \textbf{Transiently-powered swarm}
With an working communication channel, a new promising area of research could be to create a swarm of transiently powered robots, investigate the effect of intermittentcy on the behavior and controllability of this type of swarm.
This research could further investigate the portability of existing swarm algorithms, and if not possible propose new solutions.	

\item \textbf{Size reduction} 
In order to further significantly reduce the weight of a robot there is a need to move away from DC-motors, as significant smaller and efficient DC-motors are not available.
%Alternative legged locomotion types have been discussed in Section \ref{sec:rw_locomotion}.
Micro size legged robots that make use of piezoelectric actuators seem promising, but most are still in an early stage of development.

\item \textbf{Sensing capabilities}
Depending on future research or applications additional sensors could be added to the robot to extend its capabilities.
However, the power consumption, frequency of use and accuracy trade off needs to be evaluated carefully while the energy budget is limited.


\end{itemize}

%\item \textbf{TP swarm OS} 
%Recently, embedded operating systems~\cite{trenkwalder_iros_2016} and extendable programming~\cite{pinciroli_iros_2016} languages have been created to speed up the development process of swarms, removing the need to focus on low level interactions and individual behaviors.
%In addition, multiple task and checkpoint based methods have been developed to enable computation across power cycles as described in Section \ref{sec:comp_pc}.
%Merging both paradigms could help to speed up development of transiently powered swarms. 



%\subsection{Path planning based on energy availability}

%To capture the optimal amount of solar energy along the way, a map of the expected solar power can be used to compute the optimal path. To distinguish sunny or shaded two methods are proposed in \cite{plonski_tranro_2016}, one being a simple data driven Gaussian Process and the other estimates the geometry of the environment as a latent variable.
%Energy aware path planning is commonly used in combination with mission planning.
%In \cite{kaplan_iros_2016}, an analysis of the solar radiation is used to generate a time-optimized motion plan and power schedule using a cascaded particle swarm optimization algorithm.
%By combining maps of lighting and ground slope a solar-powered robot can be kept illuminated continuously. A connected component analysis is used to plan a optimal route on traversable slopes, as described by \cite{otten_icra_2015}.