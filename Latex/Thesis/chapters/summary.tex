\chapter{Summary}
\label{chp:summary}

\section{Limitations}

%TODO This is only one limitation (speed accurrracy) but there are more - please list them (lack of more sensors, communciation with the extenal world, size can be even smaller, not managed to develop a swarm).

Section \ref{sec:controlled_movements} shows that the transiently powered robot is able to execute a instructed motion with similar accuracy compared to a battery powered equal.
However, the distance covered by a robot with frequent power interrupts in a certain amount of time is smaller, i.e the average speed is lower due to more accelerations.
For this reason the robot needs a method to determine it's speed without wheel encoders.
One possibility to achieve this would be to determine motor speed by measuring the Back-EMF produced by the motor, as it is proportional to the motors revolutions per minute. % ~cite{precision_backemf_2017}.
However, the amount of power required to frequently determine the Back-EMF needs to be evaluated carefully.

Another possibility is give external feedback to the robot about its current position or control the robot's movement externally.
To achieve this the backscatter communication channel of the WISP could be used.
% ADD some info about fast downstream communication?

\section{Future Work}

This work focuses on the development of a single transiently powered robot.
A new promising area of research could be to create a swarm of transiently powered robots, and investigate the effect of intermittentcy on the behavior and controllability of this type of swarm.
This research should further investigate the portability of existing swarm algorithms, and if not possible propose new solutions.	

Recently, embedded operating systems~\cite{trenkwalder_iros_2016} and extendable programming~\cite{pinciroli_iros_2016} languages have been created to speed up the development process, removing the need to focus on low level interactions and individual behaviors.
In addition, multiple task and checkpoint based methods have been developed to enable computation across power cycles as described in Section \ref{sec:comp_pc}.
Merging both paradigms could help to speed up development of transiently powered swarms.

% Swarms restricted to light availability 

% different locomotion types?
In order to further significantly reduce the weight of a robot there is a need to move away from DC-motors, because further miniaturizing DC-motors has become a challenge.
Alternative forms of locomotion has been discussed in Section \ref{sec:locomotion}.
Micro size legged robots that make use of piezoelectric actuators have been developed but currently lack any intelligence. REF I-SWARM?

%\subsection{Path planning based on energy availability}

%To capture the optimal amount of solar energy along the way, a map of the expected solar power can be used to compute the optimal path. To distinguish sunny or shaded two methods are proposed in \cite{plonski_tranro_2016}, one being a simple data driven Gaussian Process and the other estimates the geometry of the environment as a latent variable.
%Energy aware path planning is commonly used in combination with mission planning.
%In \cite{kaplan_iros_2016}, an analysis of the solar radiation is used to generate a time-optimized motion plan and power schedule using a cascaded particle swarm optimization algorithm.
%By combining maps of lighting and ground slope a solar-powered robot can be kept illuminated continuously. A connected component analysis is used to plan a optimal route on traversable slopes, as described by \cite{otten_icra_2015}.

\section{Conclusions}
% TODO CONCLUSIONS