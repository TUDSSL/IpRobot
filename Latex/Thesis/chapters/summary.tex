\chapter{Summary}
\label{chp:summary}

This thesis is surmised by a conclusion in Section \ref{sec:conclusion} and in Section \ref{sec:limitations_future_work} possibilities for future work are presented.

\section{Conclusion}
\label{sec:conclusion}

In this thesis, a transiently-powered battery-free robot is designed and implemented.
The robot harvests energy from light and DC motors are chosen to provide locomotion.
An embedded gyroscope provides heading feedback, and a PID controller allows execution of controlled straight and curved movements.
A simple check pointing method combined with persistent movement targets allows a movement to be completed across multiple power cycles.

Straight and circular movements are executed and recorded using an overhead camera.
Tracking software is used to collect the robots movement data and compare the movement accuracy of a transiently-powered battery-free robot with its battery powered equivalent.
Using artificial power interrupts, the on time is decreased and a lower threshold of 0.3\,s is found.
The robot started to show uncontrolled behavior, i.e. significant drift to the side of the weakest motor.
On times below 0.3\,s might not allow the motors to reach their steady state speed, making linear motor control impossible.

The results show that decreasing the on time towards the lower threshold, increases the horizontal deviation in case of straight movements.
It is expected to be caused by more frequent accelerations from standstill.
For curved movements the results show a decreased average deviation from the fitted circle by increasing the target duty cycle.
The additional speed is suspected to allow the controller to be more effective in steering the robot towards its final destination.
The solar powered robot surprisingly outperforms the battery-powered robot in terms of movement accuracy.
The additional weight of the solar panel is likely to have a positive effect on the movement of the robot.

%Section \ref{sec:controlled_movements} shows that the transiently-powered robot is able to execute an instructed motion with similar accuracy compared to a battery powered equal. 
%The distance covered by a robot with frequent power interrupts in a certain amount of time is smaller, i.e the average speed is lower due to frequent acceleration from a standstill.

\section{Future Work}
\label{sec:limitations_future_work}

The capabilities of the current robot can easily be extended by using more features from the WISP or by designing a custom alternative.
Additional features that could be implemented in future work are:

\begin{itemize}

\item \textbf{Speed feedback}: 
In order to move a certain distance with a higher accuracy, an energy efficient speed feedback method is required.
One option is to add local speed sensing or another option is to supply external position feedback to the robot.

%One option could be to determine motor speed by measuring the Back-EMF produced by the motor, as it is proportional to the motors revolutions per minute~\cite{precision_backemf_2017}.
%Another method could use wheel encoders without an active light source and instead use ambient light, already available in abundance because it is used as the source of energy.

\item \textbf{Communication}: 
Communication with an external host is not implemented.
The backscatter communication channel of the WISP could be used.

\item \textbf{Transiently-powered swarm}: 
With an implemented communication channel, a promising new area of research is transiently-powered swarms.
The effect of intermittency on the behavior and controllability of this type of swarm needs further investigation.
The applicability of existing swarm algorithms is currently unknown and new methods might be required.	

\item \textbf{Size reduction}: 
In order to further reduce the weight of a robot, an alternative to DC motors has to be found.
Unfortunately, significant smaller and efficient DC motors are not available.
%Alternative legged locomotion types have been discussed in Section \ref{sec:rw_locomotion}.
Miniature legged robots that make use of piezoelectric actuators seem promising, but most of them are still in an early stage of development.

\item \textbf{Sensing capabilities}: 
Future applications may require additional sensors to be added to the robot to extend its capabilities.
However, the power consumption, frequency of use and the accuracy trade off needs to be evaluated carefully since the energy budget is limited.


\end{itemize}

%\item \textbf{TP swarm OS} 
%Recently, embedded operating systems~\cite{trenkwalder_iros_2016} and extendable programming~\cite{pinciroli_iros_2016} languages have been created to speed up the development process of swarms, removing the need to focus on low level interactions and individual behaviors.
%In addition, multiple task and checkpoint based methods have been developed to enable computation across power cycles as described in Section \ref{sec:comp_pc}.
%Merging both paradigms could help to speed up development of transiently-powered swarms. 



%\subsection{Path planning based on energy availability}

%To capture the optimal amount of solar energy along the way, a map of the expected solar power can be used to compute the optimal path. To distinguish sunny or shaded two methods are proposed in \cite{plonski_tranro_2016}, one being a simple data driven Gaussian Process and the other estimates the geometry of the environment as a latent variable.
%Energy aware path planning is commonly used in combination with mission planning.
%In \cite{kaplan_iros_2016}, an analysis of the solar radiation is used to generate a time-optimized motion plan and power schedule using a cascaded particle swarm optimization algorithm.
%By combining maps of lighting and ground slope a solar-powered robot can be kept illuminated continuously. A connected component analysis is used to plan a optimal route on traversable slopes, as described by \cite{otten_icra_2015}.