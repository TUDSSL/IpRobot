\chapter{Summary}
\label{chp:summary}

\section{Limitations and Future Work}

% Add a low power communcation method for control of multple (swarm)
% Use custom pcb with MCU + communication circuit instead of WISP

% Measure back emf to determine rpm and use it to give close loop feedback -- lowering the current peak
% Or determine the motor rpm / speed of the robot from the current
% https://www.precisionmicrodrives.com/application-notes/ab-021-measuring-rpm-from-back-emf

% Use ambient light sensor input to do path planning

% Scalable collective check part kilobot
% Not only charging but also programming and activation!

% different locomotion types?

% Combine intermittent computation with swarm operating system.
% Currently single robot and kind of checkpointing, but multi robot swarms require more advanced software?	
% OpenSwarm: An Event-driven Embedded Operating System for Miniature Robots \cite{trenkwalder_iros_2016}
% Buzz: An Extensible Programming Language for Heterogeneous Swarm Robotics \cite{pinciroli_iros_2016}

\subsection{Path planning based on energy availability}

To capture the optimal amount of solar energy along the way, a map of the expected solar power can be used to compute the optimal path. To distinguish sunny or shaded two methods are proposed in \cite{plonski_tranro_2016}, one being a simple data driven Gaussian Process and the other estimates the geometry of the environment as a latent variable.
Energy aware path planning is commonly used in combination with mission planning.
In \cite{kaplan_iros_2016}, an analysis of the solar radiation is used to generate a time-optimized motion plan and power schedule using a cascaded particle swarm optimization algorithm.
By combining maps of lighting and ground slope a solar-powered robot can be kept illuminated continuously. A connected component analysis is used to plan a optimal route on traversable slopes, as described by \cite{otten_icra_2015}.

\section{Conclusions}
% TODO CONCLUSIONS