\chapter{Evaluation} 

% Movement open loop
Open loop motion control of a robot often results in undesired drift in the position estimation.
Often being a result of unbalances or misalignment in the construction of the robot.

– Uneven wheel slip
– Uneven weight distribution
– Unbalance in driveline, motor to pcb or motor shaft → wheel

% - Weight vs current consumption vs distance
% - Measure start motor overhead and average consumption (show graph?)
% - Table with power consumption of each component at 2.2v ?? only active or also standy?

\begin{table*}[t]
	\centering
	\caption{Power consumption of each individual component at 2.2V}
	\label{tab:1}
 	\begin{tabular}{l l l} 
		\hline
		\\[-1em]
 		Part & Active Current & Standby Current\\ 
 		\hline
 		\\[-1em]
 		Proximity sensor & 432\textmu A & 0.1\textmu A \\
 		Gyroscope & 650\textmu A & 3\textmu A\\	
		Microcontroller @ 8Mhz & 1mA & N/A \\
		H-bridge & 1.2mA & 10nA \\
		DC motor & 39mA & N/A \\
		DC motor & 39mA & N/A \\
		\hline
		\\[-1em]
		Total & xx & xx \\
	\end{tabular}
\end{table*}

% outdoors/indoors/varying light sources/varying solar panels

% comparison with battery powered robot

% accuracy of motion path 

% video of robot movement