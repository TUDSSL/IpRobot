\chapter{Related Work}
\label{chp:related_work}

Background information about state-of-the-art transiently powered systems is provided in Section \ref{sec:rw_tp_systems}.
In Section \ref{sec:rw_comp_pc} methods that allow computation across power cycles are discussed.
The advantages and disadvantages of different electrical storage types are compared in Section \ref{sec:rw_energy_storage}. 
A short summary of current miniature robotics platforms is given in Section \ref{sec:rw_robotic_platforms} and commonly used locomotion types are discussed in Section \ref{sec:rw_locomotion}. 
Finally, different methods that try to ensure continuous operation are discussed in Section \ref{sec:rw_continous_operation}.

\section{Transiently-powered Systems}
\label{sec:rw_tp_systems}

% What are transiently powered systems:
Transiently-powered systems have evolved from the need to remove batteries, and instead harvest energy from ambient sources.
The ambient sources available for exploration are determined by the environment, and can be scarce or completely absent during prolonged time intervals of the day~\cite{konstantopoulos_im_2016}.
As a result, the amount of power harvested can vary significantly over time, and an energy buffer is required to guarantee the ability to complete a task.

Fully programmable RFID platforms have been developed that explore the combination of sensing, computation and communication, while allowing battery-less operation~\cite{sample_transim_2008}.
These platforms are powered from the radio signal emitted by a specialized RFID reader. 
The harvested energy is stored in a capacitor, larger capacitors can buffer more energy while smaller capacitors have the advantage of shorter charge times~\cite{gummerson_mobisys_2010}.
For longer operations that cannot be interrupted, the energy budget needs to be evaluated carefully.
To store the energy, an appropriate size storage capacitor needs to be selected, since increasing the size also increases the self discharge rate of the capacitor~\cite{naderiparizi_rfid_2015}.
The buffer still limits the operation time, resulting in frequent system power failures.

\section{Computation across Power Cycles}
\label{sec:rw_comp_pc} 
% - Short intro into persistent framwork: checkpointing etc
% TODO longer introduction + add alinea

To be able to execute long running-programs under the threat of power loss due to the energy buffer depletion, different methods have been developed to allow computation across power cycles.
There are several methods that propose solutions to computation across power cycles.
For example, programs can be automatically transformed to run under frequent loss of power by using energy-aware state checkpointing.
The state of the program is saved in non-volatile memory before running out of energy~\cite{ransford_asplos_2011}.
Another checkpointing method removes the need for special hardware or a programming model, instead it uses a compiler to add light weight non-volatile checkpoints to a program, dividing it into re-executable sections~\cite{vanderwoude_osdi_2016}.
Besides that, other research proposes a new programming model that splits a program into tasks, where the tasks exchange data trough non-volatile input and output channels, guaranteeing consistency of the program~\cite{colin_oopsla_2017}.


\section{Energy Storage}
\label{sec:rw_energy_storage}

% Three alineas:
% Why are supercapacitors so great
% Why are they not widely deployed yet?
% Solution could be capacitor / battery hybrids

% https://www.murata.com/en-eu/products/emiconfun/capacitor/2015/03/24/20150324-p1

%Capacitor vs Supercapacitor EDLC vs battery?

IoT devices are currently powered from one of two sources, either from batteries or supercapacitors.
Normal capacitors can not store enough energy to power devices and are mainly used to stabilize power supplies.
Each electrical storage technology has different properties, as seen from Table \ref{tab:cap_scap_battery}.
For example, supercapacitors have a higher power density that allows for quick charge and discharge rates without any special charging circuitry~\cite{prasad_comst_2014}.
Supercapacitors are more safe when used outside their recommended operating conditions, as they do not contain any toxic chemicals like batteries~\cite{maxwell_overview_2017}.
The two largest disadvantages of super-capacitors are their low energy density and high price.

On the other hand, batteries have a higher energy density when compared to supercapacitors and in addition, experience lower leakage currents.
A downside of using batteries is that they seldom withstand more than one thousand complete charge/discharge cycles.
Overheating of batteries can severely reduce the lifetime of a battery or in the worst case lead to explosion, which happened to the Galaxy Note 7 from Samsung in January of this year~\cite{samsung_note7_2017}.

%A combination of a li-ion battery and supercapacitor is proposed in~\cite{ongaro_pwre_2012}, and shows that the number of charge/discharge cycles of the battery can be reduced by a factor of four.


\begin{table}[t]
	\centering
	\caption{Characteristics comparison of capacitors, supercapacitors and batteries~\cite{Gonzalez_rser_2016}. Capacitors can store less energy than batteries, but have a higher power density and a larger cycle life.}
	\label{tab:cap_scap_battery}
	\small
	\begin{tabular}{|l||l|l|l|}
		\hline
		Characteristics & Capacitor & Supercapacitor & Li-ion battery \\
		\hline \hline
		Specific Energy (Wh/kg) & \textless 0.1 & 1--10 & 10--100 \\
		Specific Power (W/kg) & \textgreater 10.000 & 500--10.000 & \textless 1000 \\
		Discharge time & 10\textsuperscript{5} to 10\textsuperscript{3} & s to min & 0.3--3\,h \\
		Charge time & 10\textsuperscript{5} to 10\textsuperscript{3} & s to min & 1--5\,h \\
		Coulombic efficiency (\%) & About 100 & 85--98 & 70--85 \\
		Cycle life & Almost infinite & \textgreater 500.000 & about 1000 \\
		%Cost per Wh & \$20 (typical) & \$2 (typical) \\
		%Service life & ~10--15 years & ~5--10 years \\
		%Charge temperature & -40--65\,\textdegree C & 0--45\,\textdegree C \\
		%Discharge temperature & -40--65\,\textdegree C & -20--60\,\textdegree C \\
		\hline
	\end{tabular}
\end{table}


\section{Small Robotic Platforms}
\label{sec:rw_robotic_platforms}

% Swarm robotics requires communication to apply different algorithms
% need to be low cost and size are key factors for allowing scalability

% Scalable collective check part kilobot
% Not only charging but also programming and activation!
% Operate from batteries and some have a method of recharging
% Robots are typically evaliuated in terms of scalability and capabilities

% - Dual processor, one for motor control and one for main processing
% - Or lower level control and one for main computation

% Applications, concrete examples
% Design
% Communication and (swarm) organization


Reducing the size of robots has a number of benefits.
First of all, the amount of of materials to build a single robot is reduced, which often lowers the final production cost.
Low cost robots are, for example, developed to make them available for educational use, and allow children to come in contact with robotics and programming at early age~\cite{rubenstein_icra_2015}.

Cost and size can also be a main design considerations when developing miniature robots to research swarm behavior.
Keeping the cost down allows experiments with larger collectives of swarm robots that work together to achieve a single goal.
Hardware modularity is exploited to make robots adapt their resources to different environments and sensing operations.
By separating power, computation, motor control and sensing, a variety of capabilities can be tested~\cite{sabelhaus_icra_2013, pickem_icra_2015, kim_iros_2016}.

In order to remotely operate and/or coordinate a collective, the robots require communication with a global host accomplished by means of active low power transceivers~\cite{sabelhaus_icra_2013, pickem_icra_2015, kim_iros_2016}. 
Other microrobots use infrared-based communication, which is additionally used for neighbor to neighbor distance sensing~\cite{rubenstein_icra_2012}.
Batteries power these small robots and provide roughly one to three hours of energy.
An overview of current state of the art small robotic platforms is provided in Table \ref{tab:comparison_robot_platforms}.


\section{Locomotion}
\label{sec:rw_locomotion}

Choosing the method of actuation that allows the robot to move, i.e. locomotion type, can depend on different factors.
Moving in the most energy efficient way on a particular surface is often the determining factor.
On a flat surface, robots commonly use a two-wheeled differential drive design to not only move but allow for steering as well~\cite{sabelhaus_icra_2013, pickem_icra_2015}.
The need for wheels can be eliminated by positioning the DC motors at a 45 degree angle relative to surface and letting the motor shafts directly contact the surface~\cite{kim_iros_2016}.
A tiny ball caster is used as a third support point in the front of the robot.
Other designs do not use conventional DC motors, but instead uses stepper motors.
The motors speed can be set by changing the delay between steps. 
Estimating position is therefore reduced to simply counting steps~\cite{pickem_icra_2015}.

Another decisive factor can be the overall cost.
Vibrating motors can be combined with three thin legs~\cite{rubenstein_icra_2012}.
When the vibrating motors are activated the centripetal forces generate a forward movement, which can be explained using the slip-stick principle.
Other locomotion types are biologically inspired, a small scale piezoelectric driven quadrupled robot is designed to decrease manufacturing complexity and overall cost~\cite{baisch_iros_2013}.
%Each leg as two degrees of freedom, it can move up and down, as well as forward and backward.

\section{Continuous Operation}
\label{sec:rw_continous_operation}

%TODO add example of iros paper mobile recharging station
Typically the operation time is extended by regularly checking the remaining energy in the battery and move to a recharging station before the robot runs out of energy~\cite{pickem_icra_2015, rubenstein_icra_2012}.

An alternative to quick recharging is to swap the battery automatically when the robot moves into the docking station~\cite{kemal_mech_2015}.
An example is a robot which is able to swap it's primary battery using a six degree-of-freedom manipulator.
The manipulator is used to grab the dead battery and plug it into a wireless recharging charging station~\cite{zhang_conel_2013}.

Direct wireless power can be used as an alternative to batteries to provide power to a robot~\cite{karpelson_icra_2014}.
However, the robot can only operate or recharge while it remains in close proximity to a transmitter. 
In this case the robot is highly reliant on an infrastructure to allow for continuous autonomous operation, and can not operate in an area where this infrastructure is not present.
 
Persistent operation can be achieved by harvesting renewable energy to complement to the robots internal energy source. 
To remove weight from the robot, in~\cite{bruhwiler_iros_2015} the solar energy is used directly without any type of energy buffer. 
A drawback of this method is that the incoming solar energy should be greater or equal to the energy required for operation. 
This approach has only been tested for basic locomotion and has not combined any form of sensing or control.

% - Provide overview table robots smaller than 15*15cm

% For each of these cases you need to provide numbers: 
% level of autonomy (does the robot does all by itself or relies on external processing)
% does autonomy fall under 
% charging time

% Add missing "new" robots

\begin{table}[t]
	\centering
	%\resizebox{\columnwidth}{!}{%
	\rotatebox{90}{
		\begin{threeparttable}
			\caption{The small robotic platforms are comparable in cost, size and weight. However, it is clear that the transiently-powered robot has tiny energy capacity, very short operation time but the ability to recharge locally.}
			\label{tab:comparison_robot_platforms}
	 		\begin{tabular}{|l l l l l l l l l|} 
				\hline 
	 			Robot  & Cost & Locomotion & Speed & Size                    & Weight & Energy  & Operation & Recharge\\ 
	 			       & (\euro)&          & (cm/s)& (mm\textsuperscript{2}) & (g)       & Capacity (mAh)& Time      & Method \\ 
	 			\hline\hline
	 			This robot & 59 & wheel & 25 & 35$\times$40 & 22 & 0.006 & 1 s & solar \\
	 			Roverables \cite{dementyev_uist_2016} & 34\textsuperscript{3} & wheel & N/A & 40$\times$26 & 36\textsuperscript{3} & 100 & 45 min & inductive \\                    
	 			Zooids \cite{legoc_uist_2016}& 43 & wheel & 50 & 26$\times$26 & 12 & 100 & 1 h & manual \\      
	 			mROBerTO \cite{kim_iros_2016} & 52\textsuperscript{2} &motor shaft & 15 & 16$\times$16 & 10\textsuperscript{3} & 120 & 1.5 h & manual \\
	 			GRITSBot \cite{pickem_icra_2015} & 43\textsuperscript{2} & wheel & 25 & 31$\times$30 & 60\textsuperscript{3} & 150 & 1 h & contact \\
	 			TinyTerp \cite{sabelhaus_icra_2013} & 43 & wheel & 50 & 17$\times$18 & N/A & 50 & 1 h & manual \\
	 			Kilobot \cite{rubenstein_icra_2012} & 43\textsuperscript{2} & vibration & 1 & 33$\times$33 & 17.6\textsuperscript{3} & 160 & 3 h & manual (bulk) \\
				HAMR-VP\textsuperscript{1} \cite{bruhwiler_iros_2015}& N/A & legged & 44 & 44$\times$44 & 2.3 & 8 & 3 min & manual \\
				\hline
			\end{tabular}
			\begin{tablenotes}
				\small
				\item [1] Modified to include on-board power, sensing and control.
				\item [2] Cost of parts
				\item [3] Obtained by contacting the authors of the paper
			\end{tablenotes}
		\end{threeparttable}
	}
\end{table}