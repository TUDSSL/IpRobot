\chapter{Related Work}
\label{chp:related_work}

This chapter will provide background information about current state of the art transiently powered systems and methods that allow computation across power cycles in Section \ref{sec:rw_tp_systems} and \ref{sec:rw_comp_pc}. 
The advantages and disadvantages of different electrical storage types are compared in Section \ref{sec:rw_energy_storage}. 
A short summary of current miniature robotics platforms is given and commonly used locomotion types are discussed in Section \ref{sec:rw_robotic_platforms} and \ref{sec:rw_locomotion}. 
Finally, different methods that try to ensure continuous operation will be discussed in Section \ref{sec:rw_continous_operation}.

\section{Transiently-powered Systems}
\label{sec:rw_tp_systems}

% What are transiently powered systems:
Transiently-powered systems have evolved from the need to remove batteries, and instead harvest energy from ambient sources.
The ambient sources available for exploration are limited by the application, and can be scarce or completely absent during prolonged time intervals of the day~\cite{konstantopoulos_im_2016}.
A result of this is that the amount of power harvested can very significantly over time, and an energy buffer is required to grantee the ability to complete a task.

Fully programmable RFID platforms have been developed that explore the combination of sensing, computation and communication, while allowing battery-less operation by harvesting energy from RF~\cite{sample_transim_2008}. 
The harvested energy is stored in a capacitor, where larger capacitors can buffer more energy and smaller capacitors have the advantage of shorter charge times~\cite{gummerson_mobisys_2010}.
For longer, complex operations the energy budged needs to be evaluated carefully.
To store the energy an appropriate size storage capacitor needs to be selected, as increasing the size also increases the self discharge rate of the capacitor~\cite{naderiparizi_rfid_2015}.
However, the buffer still limits the operation time, resulting in very frequent power cycling of the system.

\section{Computation across Power Cycles}
\label{sec:rw_comp_pc} 
% - Short intro into persistent framwork: checkpointing etc
% TODO longer introduction + add alinea

To be able to execute long running-programs under the threat of power loss due to the energy buffer depletion, different methods have been developed to allow computation across power cycles.
Programs can be automatically transformed to run under frequent loss of power by using energy-aware state checkpointing, the program state is saved in non-volatile memory before running out of energy~\cite{ransford_asplos_2011}.
Another checkpointing method removes the need for special hardware or a programming model, instead it uses a compiler to add light weight non-volatile checkpoints to a program, dividing it into re-executable sections~\cite{vanderwoude_osdi_2016}.
Other research proposes a new programming model that splits a program into tasks, where the tasks exchange data trough non-volatile input and output channels, guaranteeing consistency of the program~\cite{Colin_oopsla_2017}.


\section{Energy Storage}
\label{sec:rw_energy_storage}

% Three alineas:
% Why are supercapacitors so great
% Why are they not widely deployed yet?
% Solution could be capacitor / battery hybrids

% https://www.murata.com/en-eu/products/emiconfun/capacitor/2015/03/24/20150324-p1

%Capacitor vs Supercapacitor EDLC vs battery?

IoT devices are currently powered from one of two sources, either from batteries or supercapacitors.
Normal capacitors are mainly used to stabilize the converted power from these sources.
Each electrical storage technology has different properties, as seen from Table \ref{tab:cap_scap_battery}.
For example, supercapacitors have a higher power density that allows very quick charge and discharge rates without any special charging circuitry~\cite{prasad_comst_2014}.
Supercapacitors are more safe when abused or operating under extreme conditions, as they do not contain any toxic chemicals like batteries~\cite{maxwell_overview_2017}.
The biggest disadvantages of super-capacitors is their low energy density and high price.

On the other hand, batteries have a high energy density when compared to supercapacitors and in addition batteries experience a lower leakage currents.
Overheating of batteries can severely reduce a batteries lifetime and in worst case lead to explosion, as seen back in January of this year with the Galaxy Note 7 from Samsung~\cite{samsung_note7_2017}.
%Batteries will seldom withstand more than one thousand complete charge/discharge cycles.
A combination of a li-ion battery and supercapacitor is proposed in~\cite{ongaro_pwre_2012}, and shows that the number of charge/discharge cycles of the battery can be reduced by a factor of four.


\begin{table}[t]
	\centering
	\begin{threeparttable}
		\caption{Comparison of capacitors, supercapacitors and batteries~\cite{Gonzalez_rser_2016}.}
		\label{tab:cap_scap_battery}
		\small
		\begin{tabular}{|l||l|l|l|}
			\hline
			Characteristics & Capacitor & Supercapacitor & Li-ion battery \\
			\hline \hline
			Specific Energy (Wh/kg) & \textless 0.1 & 1--10 & 10--100 \\
			Specific Power (W/kg) & \textgreater 10.000 & 500--10.000 & \textless 1000 \\
			Discharge time & 10\textsuperscript{5} to 10\textsuperscript{3} & s to min & 0.3--3\,h \\
			Charge time & 10\textsuperscript{5} to 10\textsuperscript{3} & s to min & 1--5\,h \\
			Coulombic efficiency (\%) & About 100 & 85--98 & 70--85 \\
			Cycle life & Almost infinite & \textgreater 500.000 & about 1000 \\
			%Cost per Wh & \$20 (typical) & \$2 (typical) \\
			%Service life & ~10--15 years & ~5--10 years \\
			%Charge temperature & -40--65\,\textdegree C & 0--45\,\textdegree C \\
			%Discharge temperature & -40--65\,\textdegree C & -20--60\,\textdegree C \\
			\hline
		\end{tabular}
		%\begin{tablenotes}
		%	\small
		%	\item [1] Two panels in parallel
		%\end{tablenotes}
	\end{threeparttable}
\end{table}


\section{Small Robotic Platforms}
\label{sec:rw_robotic_platforms}

% Swarm robotics requires communication to apply different algorithms
% need to be low cost and size are key factors for allowing scalability

% Scalable collective check part kilobot
% Not only charging but also programming and activation!
% Operate from batteries and some have a method of recharging
% Robots are typically evaliuated in terms of scalability and capabilities

% - Dual processor, one for motor control and one for main processing
% - Or lower level control and one for main computation

% Applications, concrete examples
% Design
% Communication and (swarm) organization


Reducing the size of robots has a number of benefits, first of all the cost to build a single robot is reduced.
Low cost robots are developed to make them available for educational use, and allow children to come in contact with robotics and programming at early age~\cite{rubenstein_icra_2015}.

On the other hand, cost and size can also be a main design considerations when developing miniature robots to research swarm behavior.
Keeping the cost down allows experiments with larger collectives of swarm robots that work together to achieve a single goal.
Hardware modularity exploited to make the robot adapt its resources to different environments and sensing operations.
By separating out power, computation, motor control and sensing a verity of capabilities can be tested~\cite{sabelhaus_icra_2013, pickem_icra_2015, kim_iros_2016}.

In order to remotely operate and/or coordinate a collective, the robots require communication with a global host accomplished by means of active low power transceivers~\cite{sabelhaus_icra_2013, pickem_icra_2015, kim_iros_2016}. 
While other microrobots use infrared-based communication, which is additionally used for neighbor to neighbor distance sensing~\cite{rubenstein_icra_2012}.
Batteries power these small robots and provide roughly one to three hours of energy.
An overview of current state of the art small robotic platforms is provided in Table \ref{tab:comparison_robot_platforms}.


\section{Locomotion}
\label{sec:rw_locomotion}
%TODO first constrain yourself to small robots

Choosing the right locomotion type can depend on different factors, moving in the most energy efficient way on a particular surface is often the determining factor.
On a flat surface, robots commonly use a two-wheeled differential drive design to not only move but allow for steering as well~\cite{sabelhaus_icra_2013, pickem_icra_2015}.
The motor shafts of the motors on the mROBerTO directly contact the surface, eliminating the need for wheels and simplifying the design.
A tiny ball caster is used as a third support point in the front of the robot~\cite{kim_iros_2016}.
The GRITSBot does not use conventional DC motors, but instead uses stepper motors the speed can be set by changing the delay between steps. 
Estimating position is therefore reduced to simply counting steps~\cite{pickem_icra_2015}.

Overall cost can be a decisive factor, therefore the Kilobot uses two vibrating motors for locomotion combined with three thin legs.
When the motors are activated the centripetal forces will generate a forward movement, which can be explained using the slip-stick principle~\cite{rubenstein_icra_2012}.
Other locomotion types are biologically inspired, the HARM-VP is small scale piezoelectric driven quadrupled robot.
Each leg as two degrees of freedom, it can move up and down, as well as forward and backward~\cite{baisch_iros_2013}.

\section{Continuous Operation}
\label{sec:rw_continous_operation}

%TODO add example of iros paper mobile recharging station
Typically the operation time is extended by regularly checking the remaining energy in the battery and move to a recharging station before the robot runs out of energy~\cite{pickem_icra_2015, rubenstein_icra_2012}.

As an alternative to quickly recharging, the battery can also be swapped automatically when the robot moves into the docking station~\cite{kemal_mech_2015}.
Another work shows a robot which is able to swap it's primary battery using a six degree-of-freedom manipulator, used to grab the dead battery and plug it into a wireless recharging charging station~\cite{zhang_conel_2013}.

Using direct wireless power to replace or supplement to a batteries energy is shown in~\cite{karpelson_icra_2014}, however the robot can only operate or recharge while remaining in close proximity to a transmitter. 
In these cases the robots are highly reliant on an infrastructure to allow for continuous autonomous operation.
This can be a severe constraint if the robot moves out of reach or needs to operate in an area where this infrastructure is not present.
 
Persistent operation can be achieved by harvesting renewable energy, particularly solar energy to complement to the robots internal energy source. 
To remove weight from the robot, in~\cite{bruhwiler_iros_2015} the solar energy is used directly without any type of energy buffer. 
A drawback of this method is that the incoming solar energy should greater or equal to the energy required for operation. 
This approach has only been tested for basic locomotion and has not combined any form of sensing or control.

% - Provide overview table robots smaller than 15*15cm

% For each of these cases you need to provide numbers: 
% level of autonomy (does the robot does all by itself or relies on external processing)
% does autonomy fall under 
% charging time

% Add missing "new" robots

\begin{table}[t]
	\centering
	%\resizebox{\columnwidth}{!}{%
	\rotatebox{90}{
		\begin{threeparttable}
			\caption{Comparison of small robotic platforms.}
			\label{tab:comparison_robot_platforms}
	 		\begin{tabular}{|l l l l l l l l l|} 
				\hline 
	 			Robot  & Cost & Locomotion & Speed & Size                    & Weight & Energy  & Operation & Recharge\\ 
	 			       & (\euro)&          & (cm/s)& (mm\textsuperscript{2}) & (g)       & Capacity (mAh)& Time      & Method \\ 
	 			\hline\hline
	 			This robot & 59 & wheel & 25 & 35$\times$40 & 22 & 0.006 & 1 s & solar \\
	 			Roverables \cite{dementyev_uist_2016} & 34\textsuperscript{3} & wheel & N/A & 40$\times$26 & 36\textsuperscript{3} & 100 & 45 min & inductive \\                    
	 			Zooids \cite{legoc_uist_2016}& 43 & wheel & 50 & 26$\times$26 & 12 & 100 & 1 h & manual \\      
	 			mROBerTO \cite{kim_iros_2016} & 52\textsuperscript{2} &motor shaft & 15 & 16$\times$16 & 10\textsuperscript{3} & 120 & 1.5 h & manual \\
	 			GRITSBot \cite{pickem_icra_2015} & 43\textsuperscript{2} & wheel & 25 & 31$\times$30 & 60\textsuperscript{3} & 150 & 1 h & contact \\
	 			TinyTerp \cite{sabelhaus_icra_2013} & 43 & wheel & 50 & 17$\times$18 & N/A & 50 & 1 h & manual \\
	 			Kilobot \cite{rubenstein_icra_2012} & 43\textsuperscript{2} & vibration & 1 & 33$\times$33 & 17.6\textsuperscript{3} & 160 & 3 h & manual (bulk) \\
				HAMR-VP\textsuperscript{1} \cite{bruhwiler_iros_2015}& N/A & legged & 44 & 44$\times$44 & 2.3 & 8 & 3 min & manual \\
				\hline
			\end{tabular}
			\begin{tablenotes}
				\small
				\item [1] Modified to include on-board power, sensing and control.
				\item [2] Cost of parts
				\item [3] Obtained by contacting the authors of the paper
			\end{tablenotes}
		\end{threeparttable}
	}
\end{table}