\chapter{Related Work}
\label{chp:related_work}

This chapter will provide background information about current state of the art transiently powered systems. The advantages and disadvantages of different electrical storage types are compared. A short summary of current miniature robotics platforms is given and commonly used locomotion types. Finally different methods that try to ensure continuous operation will be discussed.

\section{Transiently-powered Systems}
\label{sec:tp_systems}

% - Roughly everything that is powered fron a Energy harvester

Example of energy harvesting: Prolonged energy harvesting for ingestible devices~\cite{plonski_tranro_2016}
Drug delivery
\\
Converting a Plant to a Battery and Wireless Sensor with Scatter Radio and Ultra-Low Cost

%Other sources available for exploration are often limited by the application. Secondly, most sources can be scarce or completely absent during prolonged time intervals of the day as well \cite{RN15}. 

Fully programmable RFID platforms have been developed to exploring the combination of sensing, computation and communication, while allowing battery-less operation by harvesting RF energy~\cite{sample_transim_2008}.
The amount of energy collected from RF signals is very small and decreases with the distance of the device to the transmitter.
The harvested energy is typically stored in a capacitor, where larger capacitors can buffer more energy and smaller capacitors have the advantage of shorter charge times~\cite{gummerson_mobisys_2010}.
For longer, complex operations the energy budged needs to be evaluated carefully.
To store the energy an appropriate size storage capacitor needs to be selected~\cite{naderiparizi_rfid_2015}.

\section{Computation across Power Cycles}
\label{sec:comp_pc} 
% - Short intro into persistent framwork: checkpointing etc
% mementos
% chain
% ratchet

\section{Energy Supply}
\label{sec:energy_supply}

% Explain battery vs supercapacitor
% Mention current high capacitance capacitors: EDLC

% Supercapacitors are safe; forgiving if abused (do not use toxic chemicals or scarce resources (lithium))
% Allow high charge rates

% Supercapacitors used in F1 (Kers Kinetic energy recovery system)
% Maxwell proposes use for regenerative breaking!!

% Three alineas:
% Why are supercapacitors so great
% Why are they not widely deployed yet?
% Solution could be capacitor / battery hybrids

Comparing li-ion batteries with super-capacitors there are some big differences.
Supercapacitors do not need any special charging scheme and circuity for charging, except for overcharging protection.
Secondly, super-capacitors do not require any particular current profile, the energy can be stored at any rate and when the energy is required it can be extracted at any power level.
Operating a li-ion battery outside of it's recommended operating conditions can severely reduce a batteries lifetime and result in overheating or even explosion of the battery.
Batteries will seldom withstand more than one thousand complete charge/discharge cycles.
Super-capacitors used under extreme condition's, are not likely to explode but instead rupture.
While the biggest disadvantages of super-capacitors is their low energy density and high price, their lifetime is typically hundred thousands of charge/discharge cycles.

Li-Ion Battery-Supercapacitor Hybrid Storage System for a Long Lifetime, Photovoltaic-Based Wireless Sensor Network~\cite{ongaro_pwre_2012}
Reincarnation in the Ambiance: Devices and Networks with Energy Harvesting \cite{prasad_comst_2014}

% http://www.pbctechco.com/technology-platform/energy-storage/
% https://www.supercaptech.com/battery-vs-supercapacitor
% http://batteryuniversity.com/learn/article/whats_the_role_of_the_supercapacitor


\begin{table}[t]
	\centering
	\begin{threeparttable}
		\caption{Performance comparison between li-ion batteries and supercapacitors}
		\label{tab:battery_vs_supercap}
		\small
		\begin{tabular}{|l|l|l|}
			\hline
			& Supercapacitor & Li-ion battery \\
			\hline \hline
			Charge time & 1-10 s & 10-60 min \\
			Cycle life & 1 million or 30000h & 500 and higher \\
			Cell voltage & 2.3 to 2.75 V & 3.6 to 3.7 V \\
			Specific Energy (Wh/kg) & 5 & 100-200 \\
			Specific Power (W/kg) & Up to 10.000 & 1.000 to 3.000 \\
			Cost per Wh & \$20(typical) & \$2(typical) \\
			Service life & 10 - 15 years & 5 to 10 years \\
			Charge temperature & -40 to 65 \textdegree C & 0 to 45 \textdegree C \\
			Discharge temperature & -40 to 65 \textdegree C & -20 to 60 \textdegree C \\
			\hline
		\end{tabular}
		%\begin{tablenotes}
		%	\small
		%	\item [1] Two panels in parallel
		%\end{tablenotes}
	\end{threeparttable}
\end{table}


\section{Small Robotic Platforms}
\label{sec:robotic_platforms}

% Swarm robotics requires communication to apply different algorithms
% need to be low cost and size are key factors for allowing scalability

% Scalable collective check part kilobot
% Not only charging but also programming and activation!
% Operate from batteries and some have a method of recharging
% Robots are typically evaliuated in terms of scalability and capabilities

Emphasize that current robots are all battery powered

Reducing the size of robots has a number of benefits, first of all the cost to build a single robot is reduced.
Low cost robots are developed to make them available for educational use, and allow children to come in contact with robotics and programming at early age~\cite{rubenstein_icra_2015}.

Cost can be one of the main design considerations when developing miniature robots to research swarm behavior.
Keeping the cost down allows experiments with larger collectives of swarm robots that work together to achieve a single goal.


Hardware modularity is a way to make the robot adapt its resources to different environments and sensing operations.

By separating out power, computation, motor control and sensing a verity of capabilities can be tested~\cite{sabelhaus_icra_2013, pickem_icra_2015, kim_iros_2016}.

Microrobots typically use infrared-based neighbor to neighbor distance sensing and communication~\cite{rubenstein_icra_2012, pickem_icra_2015, kim_iros_2016}.

While controlling a swarm or collective is mainly accomplished by means of active low power transceivers~\cite{sabelhaus_icra_2013, pickem_icra_2015, kim_iros_2016}. 

\section{Locomotion}
\label{sec:locomotion}
%TODO tell somthing about their accuracy!

Choosing the right locomotion type can depend on different factors, moving in the most energy efficient way on a particular surface is often the determining factor.
On a flat surface, robots commonly use a two-wheeled differential drive design to not only move but allow for steering as well~\cite{sabelhaus_icra_2013, pickem_icra_2015}.
The motor shafts of the motors on the mROBerTO directly contact the surface, eliminating the need for wheels and simplifying the design.
A tiny 1/8" ball caster is used as a third support point in the front of the robot~\cite{kim_iros_2016}.
The GRITSBot does not use conventional DC motors, requiring encoders to estimate their speed. 
Instead by using stepper motors the speed can be set by changing the delay between steps. 
Estimating it's position therefore is reduced to simply counting steps~\cite{pickem_icra_2015}.  
Overall cost can be a decisive factor, therefore the Kilobot uses two vibrating motors for locomotion combined with three thin legs.
When the motors are activated the centripetal forces will generate a forward movement, which can be explained using the slip-stick principle~\cite{rubenstein_icra_2012}.
Other locomotion types are biologically inspired, the HARM-VP is small scale piezoelectric driven quadrupled robot.
Each leg as two degrees of freedom, it can move up and down, as well as forward and backward~\cite{baisch_iros_2013}.

\section{Continuous Operation}
\label{sec:continous_operation}
%Battery replenishment

%TODO include battery tosti iron anecdote
Typically the operation time is extended by regularly checking the remaining energy in the battery and move to a recharging station before the robot runs out of energy~\cite{pickem_icra_2015, rubenstein_icra_2012}.
As an alternative to quickly recharging, the battery can also be swapped automatically when the robot moves into the docking station~\cite{kemal_mech_2015}.
Another work shows a robot which is able to swap it's primary battery using a six degree-of-freedom manipulator, used to grab the dead battery and plug it into a wireless recharging charging station~\cite{zhang_conel_2013}.
Using direct wireless power to replace or supplement to a batteries energy is shown in~\cite{karpelson_icra_2014}, however the robot can only operate or recharge while remaining in close proximity to a transmitter. 
In these cases the robots are highly reliant on an infrastructure to allow for continuous autonomous operation.
This can be a severe constraint if the robot moves out of reach or needs to operate in an area where this infrastructure is not present. 
Persistent operation can also be achieved by harvesting renewable energy, particularly solar energy to complement to the robots internal energy source. 
To remove weight from the robot, in~\cite{bruhwiler_iros_2015} the solar energy is used directly without any type of energy buffer. 
A drawback of this method is that the incoming solar energy should greater or equal to the energy required for operation. 
This approach has only been tested for basic locomotion and has not combined any form of sensing or control.

% - Provide overview table robots smaller than 15*15cm

% For each of these cases you need to provide numbers: 
% level of autonomy (does the robot does all by itself or relies on external processing)
% does autonomy fall under 
% charging time

% Add missing "new" robots

\begin{table}[t]
	\centering
	\resizebox{\columnwidth}{!}{%
		\begin{threeparttable}
			\caption{Comparison of small robotic platforms}
			\label{tab:comparison_robot_platforms}
	 		\begin{tabular}{|l l l l l l|} 
				\hline 
	 			Robot & Locomotion & Size & Weight & Energy Storage & Recharge\\ 
	 			(Cost) &           &      &        & Size \& Life   & Method  \\ 
	 			\hline\hline
	 			IPR  & wheel, & 4.0 cm & 21 g & 0.006 mAh, 1 s & solar \\
	 			(\euro59) & 25 cm/s &  &      &                &       \\
	 			HAMR-VP\textsuperscript{1} \cite{bruhwiler_iros_2015}& legged, & 4.4 cm & 2.3 g & 8 mAh, 3 m & manual\\
	 			(N/A) & 1 cm/s                                                 &        &       &            &       \\
	 			Roverables \cite{dementyev_uist_2016} & wheel, & 4.0 cm & 36 g & 100 mAh, 45 m & inductive \\
	 			(\euro34) & ??                                     &        &    &               &           \\                             
	 			Zooids \cite{legoc_uist_2016}& wheel, & 2.6 cm & 12 g & 100 mAh 1 h & manual \\
	 			(\euro43)                    & 50 cm/s &       &      &             &        \\                    
	 			mROBerTO \cite{kim_iros_2016} & motor shaft, & 1.5 cm & 10 g & 120 mAh, 1.5 h & manual \\
	 			(\euro52 \textsuperscript{2}) & 15 cm/s      &        &    &               &         \\
	 			GRITSBot \cite{pickem_icra_2015} & wheel, & 3 cm & 60 g & 150 mAh, 1 h & contact \\
	 			(\euro43\textsuperscript{2}) & 25 cm/s    &      &    &                &         \\ 
	 			Kilobot \cite{rubenstein_icra_2012} & vibration, & 3.3 cm & 17.6 g & 160 mAh, 3 h & manual \\
	 			(\euro43\textsuperscript{2}) & 1 cm/s            &        &    &              & bulk \\
	 			TinyTerp \cite{sabelhaus_icra_2013} & wheel, & 1.8 cm & ?? & 50 mAh, 1 h & manual \\
	 			(\euro43)                           &50 cm/s &        &    &             &        \\
				\hline
			\end{tabular}
			\begin{tablenotes}
				\small
				\item [1] Modified to include on-board power, sensing and control.
				\item [2] Cost of parts
			\end{tablenotes}
		\end{threeparttable}
	}
\end{table}