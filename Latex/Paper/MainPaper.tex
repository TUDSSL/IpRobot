%%%%%%%%%%%%%%%%%%%%%%%%%%%%%%%%%%%%%%%%%%%%%%%%%%%%%%%%%%%%%%%%%%%%%%%%%%%%%%%%
%2345678901234567890123456789012345678901234567890123456789012345678901234567890
%        1         2         3         4         5         6         7         8

\documentclass[letterpaper, 10 pt, conference]{ieeeconf}  % Comment this line out if you need a4paper

%\documentclass[a4paper, 10pt, conference]{ieeeconf}       % Use this line for a4 paper

\IEEEoverridecommandlockouts                              % This command is only needed if 
                                                          % you want to use the \thanks command

\overrideIEEEmargins                                      % Needed to meet printer requirements.

% See the \addtolength command later in the file to balance the column lengths
% on the last page of the document

% The following packages can be found on http:\\www.ctan.org
%\usepackage{graphics} % for pdf, bitmapped graphics files
%\usepackage{epsfig} % for postscript graphics files
%\usepackage{mathptmx} % assumes new font selection scheme installed
%\usepackage{times} % assumes new font selection scheme installed
%\usepackage{amsmath} % assumes amsmath package installed
%\usepackage{amssymb}  % assumes amsmath package installed
\usepackage{threeparttable, tablefootnote}
\usepackage{tabularx}
\usepackage{flushend}
\usepackage{textcomp}

\title{\LARGE \bf
Intermittently-powered Battery-free Robot
}


\author{Koen Schaper$^{1}$ %and Bernard D. Researcher$^{2}$% <-this % stops a space
%\thanks{*This work was not supported by any organization}% <-this % stops a space
%\thanks{$^{1}$Albert Author is with Faculty of Electrical Engineering, Mathematics and Computer Science,
%        University of Twente, 7500 AE Enschede, The Netherlands
%        {\tt\small albert.author@papercept.net}}%
%\thanks{$^{2}$Bernard D. Researcheris with the Department of Electrical Engineering, Wright State University,
%        Dayton, OH 45435, USA
%        {\tt\small b.d.researcher@ieee.org}}%
}


\begin{document}

\maketitle
\thispagestyle{empty}
\pagestyle{empty}


%%%%%%%%%%%%%%%%%%%%%%%%%%%%%%%%%%%%%%%%%%%%%%%%%%%%%%%%%%%%%%%%%%%%%%%%%%%%%%%%
%\begin{abstract}

% Limit abstract to 1500 characters!

%\end{abstract}


%%%%%%%%%%%%%%%%%%%%%%%%%%%%%%%%%%%%%%%%%%%%%%%%%%%%%%%%%%%%%%%%%%%%%%%%%%%%%%%%
\section{Introduction}
% Applications
% Robots typically battery powered (write swarm rely on recharging of batteries)
% Why do we want to remove the battery?
% Add solar path planning??

Normally devices require a steady energy supply, were they operate for long periods of time without energy supply depletion. 
This continuous power model is not concerned with any kind of interrupt due to loss of energy, while it always assumes a stable energy supply that allows persistent operation.
Other devices lack the luxury of having this energy supply stability.
When this kind of device would operate under the continuous model, it could leave the device in an inconsistent state when a power interrupt occurs.
A solution could be to estimate the energy budget between every power interrupt and make sure that the execution of a task will finish before running out of energy.
However this would imply a custom implementation for each specific application. 
A more general solution is required to emulate continuous operation under a intermittent source of energy.

The intermittent power model clearly differs from the continuous power model, since it has periods with and without energy availability.
Energy harvesters create this intermittent availability of energy.
While energy is being harvested the device is off and the energy is stored in a small buffer.
When a certain energy threshold is reached the connected device is activated and able to operate for a short period of time.
In different areas, problems related to the use of the intermittent power model have already been observed.
For computation under an intermittent source of energy, a variety of solutions have been proposed.
In intermittent computation the general idea is to split a program into blocks of code, ensuring that these blocks can be re-executed and therefore keeping a consistent state of the program.
Two way communication typically implements a closed loop were the messages are validated after being send.


Using a micro-controller to control another device operating under an intermittent power supply provides new challenges.
When a power interrupt occurs the progress can be uncertain and it can be doubtful if an output operation can be redone.
In the continuous power model the assumption is made that the output is enabled for long enough to complete the operation.
But operation under the intermittent power model only two point's in time can be certain, before and after doing the output operation.
During execution of the output operation, the progress can not be determined. % gray area
Extending one period by adding multiple operations together increases the uncertainty because of a greater possible error.
%Extending the duration of a output operation could increase the possible error.

% Expain the problem of counting before or after
Counting a completed output operation is commonly done when it has been finished.
What can happen is that the output operation has been finished while the energy has been fully depleted and a completed output operation has not been counted.
This will imply that an extra unwanted output operation will be preformed.
On the other hand when output operations are counted before they are actually computed, it can happen that an operation is counted while not being finished.
As a result less operations will be counted than actually preformed.

% Simple example blinking led?

% Link to robotics

The current robotic community assumes that a stable energy supply is a requirement to allow long term persistent autonomous operation of robots.
Batteries are generally the stable energy source of choice, while different solutions have been proposed to recharge them.
Other solutions require the robot to be in close proximity to an energy source.


% Problem statement and research contribution
% 
All the previous work assumes that an operation is only possible when there is sufficient energy to complete a task.
This research will explore the feasibility of a battery-less robot, allowing persistent operation while having a very small and intermittent source of energy.
Intermittently powering robots creates new challenges in control, navigation and collaboration.

% Research question:
% has energy available for short periods of time, would it be possible to do some form of actuation?

% What are the limits of intermittently powered robots to complete a complex task autonomously?
% What are the limits of a intermittently powered device to achieve locomotion?
How to enable an intermittently powered robot to complete a complex task autonomously?

% Contributions

The contributions of the paper are as follows:

\begin{enumerate}
\item A novel battery-less robot design, were the robot purely runs of harvested energy. Which includes computation, sensing and short range communication.

\item A evaluation of the battery-less robot compared to a battery powered robot in terms of, weight, speed of movement and navigation accuracy.
\end{enumerate}
 

\section{Related Work}

\subsection{Transiently-powered systems}

Fully programmable RFID platforms have been developed to exploring the combination of sensing, computation and communication, while allowing battery-less operation by harvesting RF energy~\cite{sample_transim_2008}.
The amount of energy collected from RF signals is very small and decreases with the distance of the device to the transmitter.
The harvested energy is typically stored in a capacitor, where larger capacitors can buffer more energy and smaller capacitors have the advantage of shorter charge times~\cite{gummerson_mobisys_2010}.
For longer, complex operations the energy budged needs to be evaluated carefully.
To store the energy an appropriate size storage capacitor needs to be selected~\cite{naderiparizi_rfid_2015}.


\subsection{Batteries vs Supercaps}

"Low-hanging fruit is nowhere to be seen in fields as crucial as digital electronics, biomedical devices, or space technology."
\cite{zachary_spec_2016}

Comparing li-ion batteries with super-capacitors there are some big differences.
Super-capacitors do not need any special charging scheme and circuity for charging, except for overcharging protection.
Secondly, super-capacitors do not require any particular current profile, the energy can be stored at any rate and when the energy is required it can be extracted at any power level.
Operating a li-ion battery outside of it's recommended operating conditions can severely reduce a batteries lifetime and result in overheating or even explosion of the battery.
Batteries will seldom withstand more than one thousand complete charge/discharge cycles.
Super-capacitors used under extreme condition's, are not likely to explode but instead rupture.
While the biggest disadvantages of super-capacitors is their low energy density and high price, their lifetime is typically hundred thousands of charge/discharge cycles.

Li-Ion Battery-Supercapacitor Hybrid Storage System for a Long Lifetime, Photovoltaic-Based Wireless Sensor Network	\cite{ongaro_pwre_2012}
Reincarnation in the Ambiance: Devices and Networks with Energy Harvesting \cite{prasad_comst_2014}

% - Energy harvesting

%Other sources available for exploration are often limited by the application. Secondly, most sources can be scarce or completely absent during prolonged time intervals of the day as well \cite{RN15}. 

\subsection{Small robotic platforms}

Low cost robotic platforms have been developed to tackle a variety of challenges anonymously.
Miniature robots can be used for inspection in difficult to reach places, operating like mobile sensing units.
Hardware modularity is a way to make the robot adapt its resources to different environments and sensing operations.
By separating out power, computation, motor control and sensing a verity of capabilities can be tested~\cite{sabelhaus_icra_2013, pickem_icra_2015, kim_iros_2016}.
Microrobots typically use infrared-based neighbor to neighbor distance sensing and communication~\cite{rubenstein_icra_2012, pickem_icra_2015, kim_iros_2016}.
While controlling a swarm or collective is mainly accomplished by means of active low power transceivers~\cite{sabelhaus_icra_2013, pickem_icra_2015, kim_iros_2016}. 

\subsection{Continuous operation} 
%Battery replenishment

Typically the operation time is extended by regularly checking the remaining energy in the battery and move to a recharging station before the robot runs out of energy~\cite{pickem_icra_2015, rubenstein_icra_2012}.
As an alternative to quickly recharging, the battery can also be swapped automatically when the robot moves into the docking station~\cite{kemal_mech_2015}.
Another work shows a robot which is able to swap it's primary battery using a six degree-of-freedom manipulator, used to grab the dead battery and plug it into a wireless recharging charging station \cite{zhang_conel_2013}.
Using direct wireless power to replace or supplement to a batteries energy is shown in~\cite{karpelson_icra_2014}, however the robot can only operate or recharge while remaining in close proximity to a transmitter. 
In these cases the robots are highly reliant on an infrastructure to allow for continuous autonomous operation.
This can be a severe constraint if the robot moves out of reach or needs to operate in a area where this infrastructure is not present. Persistent operation can also be achieved by harvesting renewable energy, particularly solar energy to complement to the robots internal energy source. To remove weight from the robot, in \cite{bruhwiler_iros_2015} the solar energy is used directly without any type of energy buffer. A drawback of this method is that the incoming solar energy should already greater or equal to the energy required for operation. This approach has been tested for basic locomotion and did not combine any form of sensing and control.

\subsection{Path planning based on energy availability}

To capture the optimal amount of solar energy along the way, a map of the expected solar power can be used to compute the optimal path. To distinguish sunny or shaded two methods are proposed in \cite{plonski_tranro_2016}, one being a simple datadriven Gaussian Process and the other estimates the geometry of the environment as a latent variable.
Energy aware path planning is commonly used in combination with mission planning.
In \cite{kaplan_iros_2016}, an analysis of the solar radiation is used to generate a time-optimized motion plan and power schedule using a cascaded particle swarm optimization algorithm.
By combining maps of lighting and ground slope a solar-powered robot can be kept illuminated continuously. A connected component analysis is used to plan a optimal route on traversable slopes, as described by \cite{otten_icra_2015}.



% - Provide overview table robots smaller than 15*15cm

% For each of these cases you need to provide numbers: 
% level of autonomy (does the robot does all by itself or relies on external processing)
% does autonomy fall under 
% charging time

% Add missing "new" robots

\begin{table*}[t]
	\centering
	\begin{threeparttable}
		\caption{An comparison of small robotic platforms}
		\label{tab:1}
 		\begin{tabularx}{500pt}{l l X X l l l l} 
			\hline
 			Robot & Cost & Scalability & Sensors & Locomotion & Size [cm] & Weight [g] & Battery life \\ 
 			\hline
 			IPR & TBD & charge, program & gyro & wheel, 3cm/s & 4.0 & 15 & 1s\\
 			HAMR-VP\textsuperscript{1} \cite{bruhwiler_iros_2015} & NS & none & gyroscope, optical mouse & legged, 1cm/s & 4.4 & 2.3 & 3m \\
 			Roverables \cite{dementyev_uist_2016} & NS & charge & wheel, distance, optical encoders & wheel, ?? & 4.0 & ?? & 45m \\ 
 			Zooids \cite{legoc_uist_2016} & \$50 & ?? & position, touch & wheel, 50cm/s & 2.6 & 12 & 1-2h \\ 
 			mROBerTO \cite{kim_iros_2016} & \$60\textsuperscript{1} & program & light, range, gyro, camera, accel., compass, distance, bearing & motor shaft, 15cm/s & 1.5 & ?? & 1.5h\\
 			GRITSBot \cite{pickem_icra_2015} & \$50\textsuperscript{2} & charge, program, calibrate & distance, bearing, 3d accel., 3d gyro & wheel 25cm/s & 3 & ?? & 1-5h \\
 			Kilobot \cite{rubenstein_icra_2012} & \$50\textsuperscript{2} & charge, program & distance, ambient light & vibration, 1cm/s & 3.3 & ?? & 3-24h\\
 			TinyTerp \cite{sabelhaus_icra_2013} & \$50 & none & 3d gyro, 3d accel. & wheel, 50cm/s & 1.8 & ?? & 1h\\
			\hline
		\end{tabularx}
		\begin{tablenotes}
			\item [1] Modified to include on-board power, sensing and control.
			\item [2] Cost of parts
		\end{tablenotes}
	\end{threeparttable}
\end{table*}

\section{Design Requirements}
% - Case study of warehouse/greenhouse inspection
% - for Transienltly-powered Locomotion

Things to take into consideration:
- Trade off between charge time and operation time.
- Weight of the robot
- Power consumption of individual components
- Lower voltage decreases power consumption and allows efficient use of the energy from the supercapacitor.
- single mainboard design

% Short dutycycle charging and discharging (result of capacitor size)s
% To demonstrate intermittent operation!!


% - Accurate locomotion
% - Range measurements
% - Wireless communication with a global host




\section{Robot Design}

% - Size and weight/energy availability/speed trade-off model and 
% - Should the subsections more or less reflect the columns in the table?
% - No optical encoders because they require a lot of energy!

\subsection{Computation and Sensing}

The robot is designed around a modified WISP 5.0, utilizing the processor and backscatter communication.
It's based around a Texas Instruments MSP430FR5969 ultra low power microcontroller, operating at 16 MHz and featuring 64 KB FRAM, 2 KB SRAM and 40 IO. 

%The harvester IC is removed from the WISP in order to allow an external harvester to supply the power to the system by harvesting solar energy. 

The mainboard is the platform that connects everything together.
Most reference designs are extendable but this adds complexity, while weight and size are the main constraints of this robot design.
All the sensors can be interfaced trough a I2C connection.
To conserve as much power as possible each sensor can be enabled and disabled using dedicated pins.

For avoidance of obstacles, a Maxim Integrated MAX44000 proximity sensor was added of the robot facing forward.
This sensor can measure the amount of ambient light as well.
In addition the robot has a Bosch BMG250 Mems Gyroscope to measure yaw angle change and correct when necessary.

\subsection{Locomotion}

Two 6mm geared dc motors from Precision Microdrives are mounted in a 3d printed frame, directly under the mainboard of the IPR.
The motors are mounted diagonally opposite from each other making the robot as compact as possible.
A small wheel is mounted directly on the motor shaft and in the other diagonally opposite corners a free running wheel is mounted to the frame.
The speed of each motor can be controlled individually using a dual H-bridge, this differential drive configuration allows the robot to steer.

%The buck converter is can only supply 110mA of current.
On average the each motor consumes 38mA while running on a flat surface, which is well within the current limit that the buck converter can supply.
However when the motors are in not moving yet the inrush/start current is equal to the stall current, which is equal to 240mA for each motor.
This amount of current can not be supplied by the powersupply, using PWM to soft start the motor the average current can be limited and allowing a bulk capacitor to supply the voltage.

\subsection{Energy Harvesting}

% RF harvesting seems prommesing
% better to only use RF for communication and harvest energy from another source \cite{konstantioulos}
% wispcam long time to charge

%Photovoltaic cells used for indoor and outdoor energy harvesting, commonly have a different spectral sensitivity depending on the nature of their sources.
%Solar cells used in indoor applications need to have a high spectral sensitivity in the range of visible light (400 to 700nm).
%While for outdoor applications have a spectral sensitivity range can be broader, including near infrared  (500 nm to 1100 nm).
%Triple junction solar cells harvest in some cases up to 50\% of there energy out of the near infrared range.
% reference to paper tudelft!

% http://www.chip1stop.com/web/RUS/en/tutorialContents.do?page=009

The solar energy harvesting system is based around a Texas Instruments BQ25570. 
It includes a nano power boost charger with maximum power point tracking to extract the optimal amount of energy from the solar panel. 
% Why is this size capacitor chosen?
This harvested energy is stored in a 22mF - 4.5V supercapactor from AVX, chosen for it's low leakage current and small size.
The energy stored in supercap is a function of the capacitance and voltage difference between the plates, being equal to:

\begin{equation}
\label{eq:cap1}
E = \frac{1}{2}CV^{2}
\end{equation}

However, to be able to use the energy stored in the capacitor efficiently a voltage regular is required to supply a stable voltage to the connected loads.
The regulated output voltage is a lower threshold of the energy that can be used from the capacitor.
Lowering the output voltage allows for more energy to be used from the supercapacitor, but in general also lowers the power consumption of individual components. Rewriting Equation \ref{eq:cap1} results in Equation \ref{eq:cap2}:

\begin{equation}
\label{eq:cap2}
E = \frac{1}{2}C(V_{max} - V_{min})^{2}
\end{equation}

The Texas Insturments BQ25570 has a buck converter to efficiently regulate the capacitor voltage down to a system voltage of 2.2 Volt.

% The maximum speed that can be reached is dependent on the voltage supplied to the motors.
% The current consumed by the motors also decreases with a lower voltaged supplied. 
% The buck converter can supply up to 110mA of output current.
% Powering both motors and MCU should not exceed this current limit.

\subsection{Software}

% General library for reading and writing to i2c.

% How program consistency (ie progress) is maintained given the intermittend nature of the robot.

% A operating system for intermittend devices

\section{Evaluation} 

% Movement open loop
Open loop motion control of a robot often results in undesired drift in the position estimation.
Often being a result of unbalances or misalignment in the construction of the robot.

– Uneven wheel slip
– Uneven weight distribution
– Unbalance in driveline, motor to pcb or motor shaft → wheel

% - Weight vs current consumption vs distance
% - Measure start motor overhead and average consumption (show graph?)
% - Table with power consumption of each component at 2.2v ?? only active or also standy?

\begin{table*}[t]
	\centering
	\caption{Power consumption of each individual component at 2.2V}
	\label{tab:1}
 	\begin{tabular}{l l l} 
		\hline
		\\[-1em]
 		Part & Active Current & Standby Current\\ 
 		\hline
 		\\[-1em]
 		Proximity sensor & 432\textmu A & 0.1\textmu A \\
 		Gyroscope & 650\textmu A & 3\textmu A\\	
		Microcontroller @ 8Mhz & 1mA & N/A \\
		H-bridge & 1.2mA & 10nA \\
		DC motor & 39mA & N/A \\
		DC motor & 39mA & N/A \\
		\hline
		\\[-1em]
		Total & xx & xx \\
	\end{tabular}
\end{table*}

% outdoors/indoors/varying light sources/varying solar panels

% comparison with battery powered robot

% accuracy of motion path 

% video of robot movement

\section{Limitations and Future Work}

% Use custom pcb with MCU + communication circuit instead of WISP

% Use ambient light sensor input to do path planning

% Scalable collective check part kilobot
% Not only charging but also programming and activation!

% different locomotion types?

\section{Conclusion}


%A conclusion section is not required. Although a conclusion may review the main points of the paper, do not replicate the abstract as the conclusion. A conclusion might elaborate on the importance of the work or suggest applications and extensions. 

\addtolength{\textheight}{-12cm}   % This command serves to balance the column lengths
                                  % on the last page of the document manually. It shortens
                                  % the textheight of the last page by a suitable amount.
                                  % This command does not take effect until the next page
                                  % so it should come on the page before the last. Make
                                  % sure that you do not shorten the textheight too much.

%%%%%%%%%%%%%%%%%%%%%%%%%%%%%%%%%%%%%%%%%%%%%%%%%%%%%%%%%%%%%%%%%%%%%%%%%%%%%%%%



%%%%%%%%%%%%%%%%%%%%%%%%%%%%%%%%%%%%%%%%%%%%%%%%%%%%%%%%%%%%%%%%%%%%%%%%%%%%%%%%



%%%%%%%%%%%%%%%%%%%%%%%%%%%%%%%%%%%%%%%%%%%%%%%%%%%%%%%%%%%%%%%%%%%%%%%%%%%%%%%%
%\section*{APPENDIX}



%\section*{ACKNOWLEDGMENT}




%%%%%%%%%%%%%%%%%%%%%%%%%%%%%%%%%%%%%%%%%%%%%%%%%%%%%%%%%%%%%%%%%%%%%%%%%%%%%%%%

\bibliographystyle{IEEEtran}
\bibliography{IEEEabrv,bibtex}


\end{document}