%%%%%%%%%%%%%%%%%%%%%%%%%%%%%%%%%%%%%%%%%%%%%%%%%%%%%%%%%%%%%%%%%%%%%%%%%%%%%%%%
%2345678901234567890123456789012345678901234567890123456789012345678901234567890
%        1         2         3         4         5         6         7         8

\documentclass[letterpaper, 10 pt, conference]{ieeeconf}  % Comment this line out if you need a4paper

%\documentclass[a4paper, 10pt, conference]{ieeeconf}       % Use this line for a4 paper

\IEEEoverridecommandlockouts                              % This command is only needed if 
                                                          % you want to use the \thanks command

\overrideIEEEmargins                                      % Needed to meet printer requirements.

% See the \addtolength command later in the file to balance the column lengths
% on the last page of the document

% The following packages can be found on http:\\www.ctan.org
%\usepackage{graphics} % for pdf, bitmapped graphics files
%\usepackage{epsfig} % for postscript graphics files
%\usepackage{mathptmx} % assumes new font selection scheme installed
%\usepackage{times} % assumes new font selection scheme installed
%\usepackage{amsmath} % assumes amsmath package installed
%\usepackage{amssymb}  % assumes amsmath package installed

\title{\LARGE \bf
Intermittently-powered Battery-free Robot
}


\author{Koen Schaper$^{1}$ %and Bernard D. Researcher$^{2}$% <-this % stops a space
%\thanks{*This work was not supported by any organization}% <-this % stops a space
%\thanks{$^{1}$Albert Author is with Faculty of Electrical Engineering, Mathematics and Computer Science,
%        University of Twente, 7500 AE Enschede, The Netherlands
%        {\tt\small albert.author@papercept.net}}%
%\thanks{$^{2}$Bernard D. Researcheris with the Department of Electrical Engineering, Wright State University,
%        Dayton, OH 45435, USA
%        {\tt\small b.d.researcher@ieee.org}}%
}


\begin{document}

\maketitle
\thispagestyle{empty}
\pagestyle{empty}


%%%%%%%%%%%%%%%%%%%%%%%%%%%%%%%%%%%%%%%%%%%%%%%%%%%%%%%%%%%%%%%%%%%%%%%%%%%%%%%%
%\begin{abstract}

% Limit abstract to 1500 characters!c

%\end{abstract}


%%%%%%%%%%%%%%%%%%%%%%%%%%%%%%%%%%%%%%%%%%%%%%%%%%%%%%%%%%%%%%%%%%%%%%%%%%%%%%%%
\section{INTRODUCTION}

%- "Sectionize" the paper sub-sections to ease the reading; sections you certainly need is:
 "existing autonomous robots", 
 "methods of energy harvesting for robotic and actuation properties", 
 "types of actuation and their physical properties")

% Make people interested with the first sentence
%Intermittently powered robot research platforms, that rely on harvested power are currently non existent.

Low cost robotic platforms have been developed to tackle a variety of challenges anonymously.
Miniature robots can be used for inspection in difficult to reach places, operating like mobile sensing units.
Hardware modularity is a way to make the robot adapt its resources to different environments and sensing operations.
By separating out power, computation, motor control and sensing a verity of capabilities can be tested \cite{TinyTeRP, GRITSBot}.
Microrobots typically use infrared-based neighbor to neighbor distance sensing and communication \cite{Kilobot}.
While controlling a swarm or collective is mainly accomplished by means of active low power transceivers \cite{TinyTeRP, GRITSBot}. 

%Single motor robot 1STAR complicated autonomy and everything is a turn

Choosing the right locomotion resource can depend on different factors, moving in the most energy efficient way on a particular surface is often the determining factor.
On a flat surface, robots commonly use a two-wheeled differential drive design to not only move but allow for steering as well \cite{TinyTeRP, GRITSBot}. 
In other designs overall cost is a decisive factor, Kilobot uses two vibrating motors for locomotion.
When the motors are activated the centripetal forces will generate a forward movement \cite{Kilobot}.
The GRITSBot does not use conventional DC motors, requiring encoders to estimate their speed. 
Instead by using stepper motors the speed can be set by changing the delay between steps. 
Estimating it's position therefore is reduced to simply counting steps \cite{GRITSBot}. 

%Compare the intermittently powered robotic platform with existing work

The robotic community has a long lasting interest in persistent autonomous solutions.
Advancement in technology provides more efficient components, but mainly reduce the energy conversion efficiency and the energy required for computation.
While generally the biggest energy consumers are electric motors, required for locomotion.
The operation time is limited by the energy capacity of the robot's battery.

Typically the operation time is extended by regularly checking the remaining energy in the battery and move to a recharging station before the robot runs out of energy \cite{GRITSBot, Kilobot}.
As an alternative to quickly recharging, the battery can also be swapped automatically when the robot moves into the docking station \cite{AutoBat}.
Another work shows a robot that is able to swap it's primary battery using a six degree-of-freedom manipulator, used to grab the dead battery and plug it into a wireless recharging charging station \cite{BatSwap}.
In these cases the robots are highly reliant on there docking stations to allow for continuous autonomous operation.
This can be problematic if the robot is not able to reach it's station before running out of energy.



Other work researches persistent operation by harvesting renewable energy, particularly solar energy to complement to the robots internal energy source.
To capture the optimal amount of solar energy along the way, a map of the expected solar power can be used to compute the optimal path. To distinguish sunny or shaded two methods are proposed in \cite{SolMap}, one being a simple datadriven Gaussian Process and the other estimates the geometry of the environment as a latent variable.
Energy aware path planning is commonly used in combination with mission planning.
In \cite{PlanUGV}, an analysis of the solar radiation is used to generate a time-optimized motion plan and power schedule using a cascaded particle swarm optimization algorithm.

% Improve research statement
% Challenges in porting algorithms for localization

All the previous work assumes that a operation is only possible when there is sufficient energy to complete a task. 
This research will explore the feasibility of a battery-less robot, allowing persistent operation while having a very small and intermittent source of energy.
Intermittently powering robots creates new challenges in control, navigation and collaboration.

%\section{CONCLUSIONS}

%A conclusion section is not required. Although a conclusion may review the main points of the paper, do not replicate the abstract as the conclusion. A conclusion might elaborate on the importance of the work or suggest applications and extensions. 

%\addtolength{\textheight}{-12cm}   % This command serves to balance the column lengths
                                  % on the last page of the document manually. It shortens
                                  % the textheight of the last page by a suitable amount.
                                  % This command does not take effect until the next page
                                  % so it should come on the page before the last. Make
                                  % sure that you do not shorten the textheight too much.

%%%%%%%%%%%%%%%%%%%%%%%%%%%%%%%%%%%%%%%%%%%%%%%%%%%%%%%%%%%%%%%%%%%%%%%%%%%%%%%%



%%%%%%%%%%%%%%%%%%%%%%%%%%%%%%%%%%%%%%%%%%%%%%%%%%%%%%%%%%%%%%%%%%%%%%%%%%%%%%%%



%%%%%%%%%%%%%%%%%%%%%%%%%%%%%%%%%%%%%%%%%%%%%%%%%%%%%%%%%%%%%%%%%%%%%%%%%%%%%%%%
%\section*{APPENDIX}



%\section*{ACKNOWLEDGMENT}




%%%%%%%%%%%%%%%%%%%%%%%%%%%%%%%%%%%%%%%%%%%%%%%%%%%%%%%%%%%%%%%%%%%%%%%%%%%%%%%%

\bibliographystyle{IEEEtran}
\bibliography{IEEEabrv,bibtex}


\end{document}