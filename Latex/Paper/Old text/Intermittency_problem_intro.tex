% Introdution to intermittency problem

Normally devices require a steady energy supply, were they operate for long periods of time without energy supply depletion. 
This continuous power model is not concerned with any kind of interrupt due to loss of energy, while it always assumes a stable energy supply that allows persistent operation.
Other devices lack the luxury of having this energy supply stability.
When this kind of device would operate under the continuous model, it could leave the device in an inconsistent state when a power interrupt occurs.
A solution could be to estimate the energy budget between every power interrupt and make sure that the execution of a task will finish before running out of energy.
However this would imply a custom implementation for each specific application. 
A more general solution is required to emulate continuous operation under a intermittent source of energy.

The intermittent power model clearly differs from the continuous power model, since it has periods with and without energy availability.
Energy harvesters create this intermittent availability of energy.
While energy is being harvested the device is off and the energy is stored in a small buffer.
When a certain energy threshold is reached the connected device is activated and able to operate for a short period of time.
In different areas, problems related to the use of the intermittent power model have already been observed.
For computation under an intermittent source of energy, a variety of solutions have been proposed.
In intermittent computation the general idea is to split a program into blocks of code, ensuring that these blocks can be re-executed and therefore keeping a consistent state of the program.
Two way communication typically implements a closed loop were the messages are validated after being send.


Using a micro-controller to control another device operating under an intermittent power supply provides new challenges.
When a power interrupt occurs the progress can be uncertain and it can be doubtful if an output operation can be redone.
In the continuous power model the assumption is made that the output is enabled for long enough to complete the operation.
But operation under the intermittent power model only two point's in time can be certain, before and after doing the output operation.
During execution of the output operation, the progress can not be determined. % gray area
Extending one period by adding multiple operations together increases the uncertainty because of a greater possible error.
%Extending the duration of a output operation could increase the possible error.

% Expain the problem of counting before or after
Counting a completed output operation is commonly done when it has been finished.
What can happen is that the output operation has been finished while the energy has been fully depleted and a completed output operation has not been counted.
This will imply that an extra unwanted output operation will be preformed.
On the other hand when output operations are counted before they are actually computed, it can happen that an operation is counted while not being finished.
As a result less operations will be counted than actually preformed.

% Simple example blinking led?